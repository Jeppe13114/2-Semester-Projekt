\chapter{Samarbejdskontrakt}

\section{Generelle forventninger til gruppemedlemmer}
\begin{itemize}
	\item At folk møder til aftalt tid (møder/øvelser mm.), og hvis forhindret kan man melde det over FB-gruppen.
	\item At folk deltager aktivt til øvelserne, og ikke er bange for at tage ansvar.
	\item At man holder aftalte deadlines med hensyn til opgaver til bilen samt rapportskrivning, hvis det ser svært ud må man række ud til de andre i gruppen for at få hjælp. Ultimativt er det ens eget ansvar at færdiggøre opgaverne man er blevet tildelt til de aftalte deadlines.
	\item Det er fair at forvente, at medlemmer i gruppen møder forberedt op til øvelserne.
	\item At vi i gruppen er opmærksomme på at alle forstår de teoretiske og praktiske dele af øvelserne, så ingen sidder og falder bagud.
\end{itemize}



\section{Gruppemøder}
\begin{itemize}
	\item Som udgangspunkt afholdes gruppemøder mindst 1 gang per uge – medmindre andet er aftalt. Mandag kl 12:00 - 14:00 er fast møde tidspunkt. Et gruppemedlem kan altid foreslå ekstra gruppemøder.
	\item Der kan indkaldes til ekstra møder over vores fælles Facebook-gruppe. Laurids starter som mødeleder, og fortsætter indtil andet bestemmes. Skal Lars M. være med, indkaldes mødet over mail. 
	\item Laurids udformer i samme omgang dagsordenen. 
\end{itemize}

\section{Afbud til møder}
\begin{itemize}
	\item Man skal melde afbud til møder, så snart man ved, at man er forhindret i at deltage. Der skal ALTID meldes afbud. 
	\item Man skylder kage til gruppen, hvis man udebliver uden afbud, eller hvis man gentagende gange kommer for sent.
\end{itemize}

\section{Referater af møderne}
\begin{itemize}
	\item Dennis starter med at være referent i gruppen. Med posten som referent vil man være mødeleder ved næste gruppemøde. Referenten udsender referatet – hurtigst muligt – helst samme dag.
\end{itemize}

\section{Ledelse af gruppen}
\begin{itemize}
	\item Som udgangspunkt er det den, der var referent sidst, der er leder til mødet. Lederen er ansvarlig for at følge dagsordenen, og har ordet under mødet. Lederen har sidste ord i en eventuel uoverensstemmelse ved uddelegering af opgaver
\end{itemize}

\section{Hvornår har et problem en karakter, så vejlederen bør informeres?}
\begin{itemize}
	\item Når et problem ikke længere kan løses internt blandt parterne i gruppen, drages vejlederen ind i det.
\end{itemize}

\section{Gruppens ambitionsniveau }
\begin{itemize}
	\item Projektets funktionalitet og dokumentation af systemet har gruppens 1. prioritet. Evt. udsmykning osv. ekstra features kan arbejdes på derefter.
	\item Gruppen har som udgangspunkt et højt ambitionsniveau. Vi ønsker ikke blot at bestå, men at klare os bedst muligt. I forhold til karakter sigter vi også over gennemsnit – gerne i den bedste fjerdedel. 
\end{itemize}

\section{Omgangstonen}
\begin{itemize}
	\item Snak ordenligt til hinanden.
	\item Vi skal holde diskussionerne på et fagligt niveau i stedet for at mundhugges.
	\item Man skal forholde sig konstruktivt til et problem, og man skal (hvis muligt) tage problemet med den enkelte med henblik på løsning – inden problemet tages op på gruppen.
\end{itemize}

\section{Konsekvenser}
\begin{itemize}
	\item Som udgangspunkt vil vi selv forsøge at overkomme mindre problemer og konflikter.
	\item Kan gruppen ikke overkomme dette, vil vejlederen inddrages.
	\item Hvis én i gruppen generelt har svært ved at følge samarbejdskontrakten, (problemer med grove og gentagende overtrædelser), kan en eventuel udvisning i samarbejde med vejlederen finde sted.
\end{itemize}

\section{Validering med studienummer}
Dennis Slot Larsen: \\
Søren Søgaard Enevoldsen: \\
Søren Fomsgaard Jensen: \\
Jeppe Hansen: \\
Simon Storgaard Callesen: \\
Laurids Givskov Jørgensen: \\
Gustav Dahl Pedersen: \\
VEJLEDEREN: \\
